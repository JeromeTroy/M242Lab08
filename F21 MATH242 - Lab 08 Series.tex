\documentclass[11pt]{article}

\newcommand{\labNumber}{08}
\newcommand{\labTitle}{Series}
\usepackage{multicol}
\usepackage{amsmath,amssymb,tikz,pgfplots,url}
\usepackage[colorlinks]{hyperref}
\pgfplotsset{compat=1.14}
\usepackage%
  [%
    paperheight=11in,%
    paperwidth=8.5in,%
    top=1in,%
    bottom=1in,%
    right=1in,%
    left=1in
  ]{geometry}
\usetikzlibrary{arrows.meta}
\usepackage{menukeys}
\usepackage{enumitem}

\setlength \parindent{0pt}

%%%%%% All the tcolorboxes
%%%% Used for all those fancy boxed environments.
\usepackage[most]{tcolorbox}

\newtcblisting{codebox}[1][]{%
          listing only,
          listing options={
            style=tcblatex,
%            numbers=left,
%            numberstyle=\tiny\color{red!75!black},
            escapechar=|
          },
          coltitle=white,%
          colback=green!5!white,%
          colframe=green!75!black,%
          fonttitle=\bfseries,%
          before upper={\parindent0pt},%
          before skip=5mm,
          after skip=5mm,
          #1}

\newtcolorbox{WarningBox}[1][]{%
          coltitle=blue!80!white,%
          colback=red!20!white,%
          colframe=red!50!white,%
          fonttitle=\bfseries,%
          before upper={\parindent15pt},%
          title={Caution},
          #1,
}
\newtcolorbox{Remark}[1][]{%
          coltitle=white,%
          colback=magenta!5!white,%
          colframe=magenta!75!black,%
          fonttitle=\bfseries,%
          before upper={\parindent15pt},
          title=Remark,
          #1
}

\begin{document}

{\centering
\Large{Math242 Lab \labNumber: \labTitle}
\par}

\vspace*{5mm}

\begin{Remark}[title={Lab Instructions}]
\noindent
Complete the following lab examples and exercises.

\noindent
Make sure each question is clearly labeled and all questions are answered completely.

\noindent
\textbf{Submit both the Mathematica .nb file and a .pdf file} with the titles \vspace{.5em}

{\centering lastnameLab\labNumber.nb and lastnameLab\labNumber.pdf\par}\vspace{.5em}
\noindent 
on Canvas before the deadline.
Late submissions will not be accepted.
\end{Remark}
%%%%%%%%%%%%%%%%%%%%%%%%%%%%%%%%%%%%%%%%%%%%%%%%%%%%%%%%%%%%%%%%%%
%%%%%%%%%%%%%%%%%%%%%%%%%%%%%%%%%%%%%%%%%%%%%%%%%%%%%%%%%%%%%%%%%%
%%%%%%%%%%%%%%%%%%%%%%%%%%%%%%%%%%%%%%%%%%%%%%%%%%%%%%%%%%%%%%%%%%

\section*{Introduction}

Determining if a sum

\[
\sum_{n=0}^\infty a_n
\]

converges can be re-phrased as determining if the sequence of partial sums
\[
\left\{
\sum_{n=0}^0 a_n,
\sum_{n=0}^1 a_n,
\ldots,
\sum_{n=0}^N a_n,
\ldots
\right\}
\]
converges to a limit. Consider, for example, the infinite series
\[
S = \sum_{n=1}^\infty \frac{1}{n} = \lim_{k \to \infty} \sum_{n = 1}^k \frac{1}{n}.
\]
We can construct a table of values for the first 101 partial sums of S and see if they seem to be approaching a limit.
Instead of viewing the whole table, we will only view every 10th entry, as we did with the lab on sequences.
Below we will make a function \verb|s(n)| that computes the $n^{\textnormal{th}}$ partial sum.
\begin{codebox}
def s(n):
	# do a partial sum of the first n values of 
	# \sum 1 / k
	k_values = list(range(1, n))
	partial_sum = sum([1 / k for k in k_values])
	return partial_sum

def print_table_sum(n_values, sum_values, step):
	# helper function which prints a pretty table
	print("n \t sum to n")
	print("----------------------")
	for index in range(0, len(n_values), step):
		print(n_values[index], "\t", sum_values[index])
\end{codebox}

Another way to see if a sequence may be converging or diverging is to plot the sequence of partial sums using {\texttt ListPlot}. We can do this with the first 101 terms of the above sequence:
\begin{codebox}
\# only put imports at the beggining of a python file!
import matplotlib.pyplot as plt

# plot
plt.plot(n_values, sum_values, ".")

# pretty it up
plt.xlabel("n")
plt.ylabel("s(n)")

# display the plot
plt.show()
\end{codebox}
Does this plot look like a familiar function?
Based on the plot, does the sequence of partial sums appear to be converging or diverging?

Python alone does not have the tools to determine if the sum
converges exactly.  
The \texttt{Sympy} library has the tools needed to do this,
and gives Python the same capabilities as Mathematica
\begin{codebox}
\# Again, only have imports at the top of a python file
# this imports the symbolic k, which can be used as a variable
from sympy.abc import k
# import summations, and oo is infinity
from sympy import Sum, oo

a_k = 1 / k
# create an expression for the sum, notice the capital "S" in "Sum"
sum_expression = Sum(a_k, (k, 1, oo))
# doit() actually performs the operation through computation
print(sum_expression.doit())

\end{codebox}
which gives us "oo," indicating the series goes to infinity, 
and therefore does not converge.

To see everything put together, see the python file \texttt{template.py}.
To install packages in the Thonny IDE, see
\url{https://www.youtube.com/watch?v=Oo-B98WWre_8}.

\newpage
\section*{Lab Questions}
\begin{Remark}[title={General Debugging Tip}]
	\noindent
	If your computer is taking too long and you want to cancel a calculation, from the menu bar follow, simply push "Ctrl + C." In Python this terminates
	a command which is taking too long.
\end{Remark}

\subsection*{Question 1(a)}

For each of the following series, perform the steps outlined below.
\[
\sum_{k=1}^\infty \frac{1}{k^2},
\qquad\qquad
\sum_{k=1}^\infty \left(\frac{\ln(k)}{k}\right)^2,
\qquad\qquad
\sum_{k=0}^\infty \frac{1}{k!}
\]
\vspace{.5em}

\begin{tabular}{c}
	%\hline
	\parbox{.97\textwidth}{\begin{multicols}{2}
			\begin{enumerate}[label=\roman*)]
				\item Define a sequence of partial sums. 
				\item Make an abridged table of values.
				\item (List)Plot the partial sums.
				\item Evaluate the sum of the series. \columnbreak
				\item Considering the sequence of terms $a_k$ so that the series is $\displaystyle\sum_{k}^{\infty}a_k$, determine if $a_k$ converges or diverges, and if it converges, find its limit. (Note: you can answer this question by your result in the previous step! Why?)
			\end{enumerate}
	\end{multicols}}
	%\\ \hline
\end{tabular}
\vspace{.5em}

\begin{WarningBox}[title={The Journey of an Infinite Sum Begins with a Single Index . . .}]
	\noindent
	Make sure you are careful with the starting value of the last series--the sum starts at zero instead of one.
\end{WarningBox}

\subsection*{Question 1(b)}

Take your best guess at these questions (no right or wrong answers here!).

\begin{enumerate}[label=\roman*)]
	\item Looking at the plots and lists, which series appears to be converging most quickly? 
	
	\item Which one appears to be converging most slowly? 
\end{enumerate}

\subsection*{Question 2}

Let us look more carefully at the last series in Question (1). You should have found that the series converged to $e \approx 2.718$. This is a well-known series we will revisit later in the semester. Use the sample code below as a starting point to answer the questions.
\vspace{.5em}


\begin{enumerate}[label=\roman*)]
	\item Modify the code in \verb|sum_errors.py|. It helps you to 
	compute an error list for a partial sum against a known exact value
	for the infinite sum.
	\item Use the command \verb|plt.loglog| to plot the errors on a $\log-\log$ plot. What do you notice?
	
	\item Does evaluating the partial sums seem to be a good way to find an approximation to $e$?
	
\end{enumerate}

\subsection*{Question 3}

\begin{enumerate}[label=\roman*)]
	\item Repeat Question 2 with the series
	\[
	\sum_{k=1}^\infty \frac{1}{k^2}.
	\]
	The sum of this series is known to be $\displaystyle \frac{\pi^2}{6}$, so this should be used as your value for \verb|exact|.
	
	It actually takes a lot of effort to prove the series converges to this value (you can read about the Basel Problem on Wikipedia if you are interested---and it illustrates why you are only asked to determine convergence or divergence for most series, not their limits).
	
	\item Does this series converge as quickly as the one in Question 2? Make sure your sum starts at $k = 1$ instead of $k = 0$.
\end{enumerate}

\end{document}
